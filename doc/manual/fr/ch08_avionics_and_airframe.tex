% !TeX encoding = utf8
% !TeX spellcheck = fr

\chapter{Avionique et cellule d'aéronef}\label{cha:avionics-airframe}

\begin{framed}
	\begin{center}
		\xc{} est entièrement développé par des bénévoles.\\
		Cette documentation aussi.\\
		Si vous y voyez des imperfections, vous pouvez facilement les faire disparaître~:\\
		\xcsoarwebsite{/develop/}
	\end{center}
\end{framed}

Ce chapitre considère \xc{} en tant que sous-système de l'aéronef.
Il couvre l'intégration d'\xc{} avec des périphériques externes tels que les GPS, les interrupteurs, les capteurs, les récepteurs radios et les autres matériels, à l'exception de l'intégration avec le FLARM (qui est abordée dans le chapitre~\ref{cha:airspace}) et l'intégration avec des variomètres (abordée dans le chapitre~\ref{cha:atmosph}).

\section{Autonomie de la batterie}

La plupart des terminaux personnels sont conçus pour une utilisation de courte durée, et ainsi n'ont pas une batterie de grande autonomie par rapport à la durée des vols en planeur sur la campagne.
Il est donc recommandé que les terminaux personnels disposent d'une alimentation externe, à travers un convertisseur de tension adapté, connecté à la batterie du planeur.
Cette installation doit être effectuée par une personne disposant des qualifications adéquates, et doit inclure un fusible et un interrupteur d'isolement manuel.

La plus grande source de consommation électrique d'un terminal personnel provient du rétro-éclairage de l'écran LCD.
Comme la plupart des terminaux personnels du marché ne sont pas particulièrement lumineux, ils nécessitent que leur rétro-éclairage soit réglé au maximum.
Au contraire, pour les systèmes d'instruments électroniques de vol (``EFIS''), il est recommandé de régler le rétro-éclairage à l'intensité minimale permettant une lecture confortable.

Lorsqu'un terminal personnel fonctionne sur sa batterie interne, \xc{} est capable de détecter un niveau de batterie faible et permet au système d'exploitation de s'éteindre et de préserver la mémoire.
De plus, il peut être paramétrer pour éteindre l'écran après une période d'inactivité afin de réduire la consommation électrique.
Quand l'écran est éteint, appuyer sur l'un des boutons du terminal réactive l'écran.
Quand un message d'état est publié par le système, l'écran s'active aussi.

Une autre façon de préserver la batterie est de réduire le charge de calcul en éteignant certaines fonctions.
L'affichage du relief et de traces longues contribuent significativement à la charge du CPU

Pour d'autres plateformes embarquées, une InfoBoxe \infobox{Batterie} est disponible et affiche le durée de batterie restante ainsi que l'état de charge (Alimentation branchée --- charge en cours ou Alimentation débranchée --- utilisation de la batterie interne).

\section{Connexion GPS}

\xc{} nécessite un positionnement GPS en 3~dimensions pour ses fonctions de navigation.

\subsection*{Statut du GPS}

Les icônes de statut du GPS et du texte peuvent s'afficher sur le bord inférieur de la carte pour indiquer~:

\begin{tabular}{c c}%{c c}
\includegraphics[angle=0,width=0.75cm,keepaspectratio='true']{icons/gps_acquiring.pdf} & \includegraphics[angle=0,width=0.75cm,keepaspectratio='true']{icons/gps_disconnected.pdf}\\
(a) & (b)
\end{tabular}

\begin{description}
\item[En attente d'une positino GPS (a)]  Une meilleure réception ou une durée supplémentaire pour trouver des satellites est nécessaire.
	Le GPS peut avoir une position en 2~dimensions.
	Le symbole de l'aéronef disparaît tant qu'il n'y a pas de position en 3~dimensions.
\item[GPS non connecté (b)] Aucune communication en provenance du GPS n'est reçue.
	Cela indique soit une erreur dans le paramétrage du port de communication, soit que le boîtier GPS est déconnecté ou éteint.
\end{description}

Quand le GPS ne s'est pas connecté pour plus d'une minute d'attente, \xc{} tente automatiquement de redémarrer une communication avec le boîtier et se remettra à attendre.
Cette méthode fournit le moyen le plus fiable pour récupérer à la suite d'erreurs de communication.

\xc{} peut gérer des sources GPS multiples et les utilisent afin de disposer de redondance.
Ces sources sont paramétrées dans l'écran de configuration du système, sur la page intitulée ``Périphériques''.
Le périphérique~A est la source primaire de données GPS, et le périphérique~B la source secondaire.

En fonctionnement, si la source GPS primaire est perdue, \xc{} utilisera les données provenant de la source secondaire.
Si les deux sources fournissent des positions valides, la source secondaire est ignorée.
Pour cette raison, il est recommandé de configurer comme source primaire (c.-à-d. comme périphérique~A) la source GPS ayant la meilleure antenne ou le fonctionnement le plus fiable.

\subsection*{Altitude GPS}

Certains boîtiers GPS anciens (mais aussi certains récents) ne fournissent pas une altitude par rapport au niveau moyen de la mer, mais par rapport à l'ellipsoïde WGS84.
\xc{} le détecte et applique le décalage entre l'éllipsoïde et le géode sur la base de données d'un tableau interne ayant une résolution de deux degrés.
Ce n'est pas nécessaire pour les boîtiers FLARM, qui fournissent bien une altitude au-dessus du niveau moyen de la mer.


\section{Entrées par interrupteurs et capteurs}

\xc{} permet le suivi des interrupteurs et des capteurs connectés au terminal.
Cela permet d'avoir un retour sur la situation, des alertes et de disposer de périphériques d'entrée génériques pour l'utilisateur.
Plusieurs mécanismes sont disponibles pour interfacer des interrupteurs et des capteurs~:
\begin{description}
\item[Port série] Certains variomètres intelligents tel le Vega de Triadis Engineering possèdent plusieurs interrupteurs et transmettent leurs informations vers le terminal personnel ou l'EFIS sous forme de messages NMEA spéciaux.
\item[Port Bluetooth] De nombreux terminaux supportent une connexion sans fil vers un périphérique Bluetooth telle un manette de jeu disposant de plusieurs boutons.
C'est plus approprié pour les périphériques d'entrée utilisateur que pour le suivi d'état de l'aéronef.
\end{description}

Un fichier d'`évènements d'entrées' modifiable détermine le traitement des entrées provenant des interrupteurs et des capteurs.

Des interrupteurs habituels sur l'aéronef sont~:
\begin{itemize}
\item les aérofreins
\item la position des volets (positive/atterrissage, neutre, négative/action)
\item le train d'atterrissage.
\end{itemize}

Il est prévu d'inclure aussi les suivis du moteur et du carburant.

D'autres entrées logiques venant du Vega incluent les quantités calculées en relation avec des alertes spécifiques à l'aéronef et des avertissements sur l'enveloppe de vol de l'aéronef, tel que ``aérofreins sortis et train d'atterrissage rentré''.  

Pour plus de détails sur les interrupteurs d'entrée et sur la façon de les utiliser, consultez la documentation du Vega.

\section{Fenêtre de dialogue des interrupteurs du Vega}

Une fenêtre de dialogue affichant les états des interrupteurs pour le variomètre Vega est disponible dans le menu.
\menulabel{\bmenug{Config 3}\blink\bmenug{Vega}\\
\bmenut{Interrupteurs}{aéronef}}

Cette fenêtre de dialogue est mise en jour en temps réel, permettant au pilote de vérifier le bon fonctionnement des interrupteurs lors des tests d'inspection journalière ou d'avant décollage.

\begin{center}
\includegraphics[angle=0,width=0.7\linewidth,keepaspectratio='true']{figures/dialog-switches.png}
\end{center}

\section{Mode esclave}

Dans les paramètres de configuration, le type de périphérique ``NMEA Out'' est défini pour l'utilisation simultanée de deux systèmes en mode maître-esclave.
Dans le périphérique maître, le second périphérique de communication peut être paramétré comme ``NMEA Out''.
Toutes les informations reçues du premier périphérique de communication (ainsi que les informations sortantes) seront alors envoyées à l'esclave.

Voici un exemple où deux périphériques sont connectés ensemble~: dans l'esclave, le premier périphérique de communication est paramétré comme ``Vega''. Ainsi, ce système recevra toutes les données comme si elles venaient du GPS du maître et des instruments connectés (Vega, FLARM, etc.).

\section{État du système}\label{sec:system-status}

La fenêtre de dialogue d'état du système est
\menulabel{\bmenug{Info 2}\blink\bmenug{États}}
utilisée en premier pour vérifier les systèmes, et voir comment l'ordinateur de vol et les périphériques connectés se comportent.
On y accède par le menu puis en sélectionnant le tableau \textbf{Système}.
\sketch{figures/status-system.png}

Toutes les valeurs dynamiques (par ex. la tension de la batterie, le nombre de satellites en vue) sont mis à jour temps réel.

\section{Périphériques multiples}

Vous pouvez configurer de multiples périphériques externes connectés simultanément (peu de terminaux personnels disposent de deux ports série, mais le Bluetooth peut gérer un très grand nombre de connexions simultanées).

Quand deux périphériques fournissent simultanément une position GPS valide, le premier (c.-à-d. le périphérique A) est choisi par \xc, et la position GPS du second périphérique est alors ignorée.
Dès que le premier périphérique lâche, \xc{} bascule automatiquement sur le second, et ce jusqu'à ce que le premier refonctionne.

Ceci est vrai pour toutes les valeurs (altitude barométrique, variomètre, vitesse air, trafic, etc.)~: \xc{} favorise le premier périphérique, et n'utilise le second que pour les valeurs qui ne sont pas reçues depuis le premier.

Par exemple~: le périphérique~A est un CAI~302 et le périphérique~B est un FLARM. 
Cela permet d'avoir le meilleur des deux~: \xc{} dispose alors de la vitesse air, du variomètre et des informations sur le trafic.

\section{Gestion des périphériques externes}

La fenêtre de dialogue de gestion des périphériques est accessible à partir du menu de configuration.
\menulabel{\bmenug{Config. 1}\blink\bmenug{Périph.}}
Elle affiche la liste des périphériques externes configurés et les informations qu'ils fournissent.

\config{comdevices}
Le bouton ``Reconnecter'' tente de rétablir la connexion avec le périphérique sélectionné.
\xc{} tente périodiquement de se reconnecter à un périphérique défaillant, mais, parfois, il peut être opportun de le déclencher manuellement.

Le bouton ``Télécharger le vol'' est uniquement disponible pour les enregistreurs IGC supportés (voir §~\ref{sec:supported-varios} pour la liste).
Après avoir cliqué, \xc{} récupère la liste des vols et vous demande d'en sélectionner un. 
Le fichier IGC sera téléchargé dans le répertoire ``\texttt{logs}'' de \texttt{XCSoarData}.

Le bouton ``Gérer'' est activé quand un Vega ou un CAI~302 est connecté.
Il permet d'accéder à des fonctions spécifiques de ces périphériques, tel que l'initialisation de la mémoire du CAI~302, ce qui est parfois nécessaire pour contourner un bug du firmware de Cambridge.
