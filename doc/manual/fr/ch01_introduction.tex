% !TeX encoding = utf8
% !TeX spellcheck = fr

\chapter{Introduction}\label{cha:introduction}
Ce document est le manuel d'utilisation d'\xc, logiciel de navigation aérienne, à source ouverte, développé à l'origine pour les Pocket PC. Le lecteur est censé avoir de bonnes connaissances des fondamentaux de la théorie du vol à voile et un minimum de pratique du vol sur la campagne.

Les mises à jour régulières du logiciel peuvent rendre ce manuel périmé sur certains points. Il est souhaitable de lire les notes de mise à jour du logiciel pour en connaître les évolutions. Les mises à jour du logiciel et de la documentation sont disponibles sur~:
\begin{quote}
\xcsoarwebsite{/download/}
\end{quote}

\begin{framed}
	\begin{center}
		\xc{} est entièrement développé par des bénévoles.\\
		Cette documentation aussi.\\
		Si vous y voyez des imperfections, vous pouvez facilement les faire disparaître~:\\
		\xcsoarwebsite{/develop/}
	\end{center}
\end{framed}

\section{Organisation du manuel}

\todonum[inline]{Write about the manual crossref hinting icons and the yellow
colour. The Quickstart will be readable also without those links available} 
Ce manuel est principalement écrit pour permettre à un utilisateur d'\xc{} de s'en servir
rapidement, \emph{mais aussi} de permettre une compréhension approfondie de toutes ses fonctionnalités,
concepts et tactiques introduites. Les auteurs ont fait un effort constant
pour le rédiger du point de vue du pilote (et espère sincèrement y avoir réussi).

Les auteurs vous encouragent fortement à prendre le temps de lire l'ensemble du manuel,
chapitre par chapitre (excepté pour les chapitres de références sur les Infoboxes et 
la configuration). Soyez assuré que le temps que vous y aurez passé sera profondément
utile à votre compréhension. Lors de votre lecture, vous pourriez vous sentir parfois un peu perdu.
C'est pour vous aider à vous y retrouver que les auteurs ont introduit des liens et des icônes.

\begin{figure}[h]
\centering
\includegraphics[width=0.8cm,angle=0,keepaspectratio='true']{figures/config.pdf}
\hspace{1.5cm}
\includegraphics[width=0.8cm,angle=0,keepaspectratio='true']{figures/reminder.pdf}
\hspace{1.5cm}
\includegraphics[width=0.8cm,angle=0,keepaspectratio='true']{figures/gesture.pdf}
\hspace{1.5cm}
\includegraphics[width=0.8cm,angle=0,keepaspectratio='true']{figures/warning.pdf}
\caption{Icônes de configuration, de rappel, de saisie tactile et de mise en garde}
\end{figure}

\warning Mise en garde. L'icône de mise en garde est utilisé à chaque fois que quelque chose
doit être strictement suivi. Ne pas l'effectuer pourrait amener à des résultats imprévus, un dysfonctionnement complet ou même un danger pour votre vie. Poursuivez plus avant uniquement si vous avez pleinement compris la mise en garde.

\gesture{BH} Saisie tactile. Une saisie tactile est possible sur les appareils ayant
un écran tactile, et permet d'ouvrir un menu, de lancer une action, etc. Dans le présent exemple BH
signifie ``déplacer votre doigt vers le bas, puis vers le haut (en ligne droite) sur l'écran''.
 
\gesturespec{DU} Saisie tactile spécifique. Lorsque les auteurs du manuel suivent le processus de développement rapide d'\xc{} par écrit, un icône spécifique est fourni, donnant les mouvements.

\tip Rappel. Cette icône signale un conseil, une astuce ou des choses que vous auriez besoin de vous 
rappeler après avoir lu les paragraphes correspondants, etc.

\config{orientation} Voir configuration\textellipsis Cette icône représentant deux outils
de mécanicien vous renvoie à une description approfondie des points qui sont discutés et à la
manière de les configurer. Les nombres à côté de l'icône font référence au paragraphe
des chapitres~\ref{cha:infobox} et~\ref{cha:configuration} de ce manuel. Et dans l'exemple présent,
au paragraphe~\ref{conf:orientation}. 

\marginlabel{\parbox{1.3cm}{\rotatebox[origin=c]{180}{\includegraphics[width=0.9cm]{figures/warning.pdf}}}}
\rotatebox[origin=c]{180}{Arrêtez-vous de lire les manuels lorsque vous êtes en vol dos~!}

\emph{Lisez} à la maison et \emph{configurez} au sol, en sécurité. Ayant compris cette mise en garde
(renversée) comme un exemple, vous êtes prêt à continuer.

\config{usingxcsoarsafely} Ceci est un second exemple de l'icône ``configuration''. L'icône renvoie
au chapitre~\ref{cha:introduction} (ce chapitre), paragraphe 
\ref{sec:usingxcsoarsafely}, ``Utiliser \xc{} en sécurité'' ci-dessous, qui pourrait être 
compris comme ``comment vous configurez vous-même.'' C'est à vous de décider de poursuivre
par une lecture approfondie puis de revenir en arrière, ou simplement de continuer. Si vous lisez ce manuel
sur un écran, cliquer sur le numéro vous conduira vers le paragraphe
mentionné. Utilisez la fonction ``retour'' (ou similaire) de votre
logiciel de lecture pour revenir au chapitre que vous lisiez.

Lorsque des nombres sont imprimés en bleu, comme à propos des icônes mentionnées ci-dessus, cela veut dire ``aide disponible''.
De même lorsque des adresses internet (URL) sont écrites en bleu. 
Cliquer sur un texte comme \xcsoarwebsite{/contact/} ouvrira votre navigateur web
ou votre logiciel de messagerie pour vous proposer, respectivement, d'autres ressources, ou le contact
de personnes compétentes.

Le reste de ce chapitre ``Introduction'' est là pour vous préparer à \xc,
à augmenter votre niveau de compréhension et à maintenir vos compétences.
Si vous êtes un lecteur pressé, le chapitre~\ref{cha:installation} ``Installation'' pourrait être votre prochain point de virage, suivi du chapitre~\ref{cha:quickstart} ``Prise en main''. Ne vous gênez pas pour lire
en diagonale, mais ne soyez pas non plus trop déprimé en lisant chapitre par chapitre. Le contenu des chapitres successifs est le suivant~:

Le chapitre~\ref{cha:interface} décrit les concepts de l'interface utilisateur et donne une vue générale de l'affichage.

Le chapitre~\ref{cha:navigation} décrit en détail l'utilisation de la carte mobile et l'aide que peut apporter le logiciel de navigation. Le chapitre~\ref{cha:tasks} décrit comment les circuits sont définis et utilisés en vol. Il présente aussi les outils d'analyse permettant aux pilotes d'améliorer leurs performances. Le chapitre~\ref{cha:glide} est consacré au calculateur de vol d'\xc{} et présente en détail les fonctionnalités qu'il offre. Il est important pour les pilotes de comprendre comment le calculateur effectue ses différents calculs.

Le chapitre~\ref{cha:atmosph} parle de l'interfaçage du calculateur avec des varios et autres instruments de mesure et comment ces données sont utilisées pour représenter différents modèles concernant entre autres le vent et la convection. Le chapitre~\ref{cha:airspace} parle de la gestion des espaces aériens et des alarmes dédiées ainsi que des alarmes du \fl. Le chapitre~\ref{cha:avionics-airframe} présente l'intégration du calculateur avec le reste des systèmes utilisés dans l'environnement de vol (terminaux permettant de communiquer avec le calculateur, switches divers) et des diagnostics possibles.

La suite du document est constituée principalement de chapitres de référence. Le chapitre~\ref{cha:infobox} liste les différentes informations qui peuvent être affichées dans les ``InfoBox'' sur les côtés de la carte mobile. La configuration du logiciel est décrite dans le chapitre~\ref{cha:configuration}. Le format des fichiers utilisés ainsi que la manière de les obtenir ou les créer et les éditer sont décrits dans le chapitre~\ref{cha:data-files}.

Enfin, un bref historique et une explication du processus de développement d'\xc{} sont présentés dans le chapitre~\ref{cha:history-development}.

\section{Remarques}

\subsection*{Copies d'écran}
Tout au long de ce manuel des copies d'écran d'\xc{} sont présentées. Elles proviennent de différentes plateformes matériel et pas forcément de la même version d'XCSoar. Entre plateformes, il peut y avoir des différences de résolution d'écran, de présentation générale, de police de caractères. Cela peut induire des différences entre la documentation et la visualisation sur votre appareil. La plupart des copies d'écran de ce manuel sont faites avec un affichage d'\xc{} en format paysage.

\section{Platformes}
\begin{description}
\item[Appareils Android]
\xc{} tourne sous Android~1.6 ou supérieur.
\item [eBookreader]
\xc{} tourne sur les plusieurs lecteurs Kobo.
\item[PC Windows]
\xc{} fonctionne sur un PC sous Windows (XP, Vista, 7 versions 32 et 64 bits). Cette version est principalement utile pour la prise en main, l'entraînement à l'utilisation d'\xc, rejouer des fichiers IGC enregistrés ou utiliser \xc{} en mode simulation sur un PC non connecté à un GPS.
\item[PC Unix/Linux]
\xc{} fonctionne sous Unix/Linux.
\end{description}

\section{Aide technique}

\subsection*{Dépannage}
\xc{} est développé par une petite équipe. Bien que nous soyons heureux de vous aider dans 
l'utilisation de notre logiciel, nous ne pouvons donner de cours sur l'utilisation des techniques
modernes d'information~! Si vous avez une question à propos d'\xc{} et que vous n'avez pas 
trouvé de réponse dans ce manuel, n'hésitez pas à nous contacter.
Vous trouverez l'ensemble des contacts résumés ici~:
\begin{quote}
	\xcsoarwebsite{/contact}
\end{quote}

Pour démarrer, allez sur le forum XCSoar~:
\begin{quote}
	\url{https://forum.xcsoar.org}
\end{quote}
Si votre question n'y a pas déjà trouvé de réponse, postez-y, ou envoyez-nous un courriel~:
\begin{quote}
	\href{mailto:xcsoar-user@lists.sourceforge.net}{xcsoar-user@lists.sourceforge.net}
\end{quote}
Toute question fréquente sera ajoutée à ce document.
Vous pourriez aussi trouver utile de vous abonner à la liste de diffusion des utilisateurs
d'\xc{} afin d'être tenu au courant des derniers développements.

Si aucun de ces moyens ne vous aide, vous êtes probablement tombé sur un bug.

\subsection*{Vos retours}
Comme tout logiciel élaboré, \xc{} peut comporter des bugs. Si vous en trouvez un,
veuillez en informer à l'équipe de développement en utilisant notre portail dédié sur~:
\begin{quote}
\xcsoarwebsite{/develop/new_ticket.html}
\end{quote}
ou en nous contactant par mail à~:
\begin{quote}
\href{mailto:xcsoar-devel@lists.sourceforge.net}{xcsoar-devel@lists.sourceforge.net}
\end{quote}
\xc{} enregistre beaucoup d'informations utiles dans un fichier de log
\verb|xcsoar.log| situé dans le répertoire \texttt{XCSoarData}. Le fichier de log peut être attaché
au ticket de bug afin d'aider les développeurs d'\xc{} à déterminer la cause possible du problème.

Et si vous aimez l'idée d'en faire un peu plus, N'hésitez pas à contribuer~:
\begin{quote}
\xcsoarwebsite{/develop/}
\end{quote}

\subsection*{Mises à jour}
Vous devriez périodiquement aller sur le site web d'\xc{} afin de vérifier s'il
y a des mises à jour du logiciel. La procédure d'installation décrite dans le chapitre suivant peut typiquement
être répétée afin de mettre à jour le logiciel. Tous les paramètres de configuration
de l'utilisateur et les fichiers de données seront conservés au cours
d'une ré-installation ou d'une mise à jour.

Il est aussi recommandé de vérifier périodiquement s'il y a des mises à jour des fichiers
de données, spécialement pour les espaces aériens qui peuvent avoir subi des 
changements par l'autorité nationale en charge de l'aviation civile.

\section{Entraînement}
Pour votre sécurité et celle des autres, les pilotes utilisant \xc{} doivent s'entraîner \emph{au sol}
à l'utilisation du logiciel, et ainsi s'habituer à l'interface utilisateur et aux différentes fonctionnalités qu'il offre, et ce AVANT de l'utiliser en vol.

\subsection*{\xc{} sur un PC}
La version PC permet de se familiariser avec le logiciel, son interface utilisateur et ses fonctionnalités tout en étant confortablement installé à la maison, une bière à la main\textellipsis Tous les fichiers et les configurations de cette version sont identiques aux versions embarquées. Il est donc très facile de tester différentes configurations sur le PC avant de les mettre en pratique en vol.

La version PC peut être connectée à des instruments et fonctionner comme la version embarquée. Exemples d'utilisation~:
\begin{itemize}
\item connecter un \fl{} au PC pour utiliser \xc{} comme station au sol, et alors afficher le trafic des planeurs équipés de \fl.
\item connecter un variomètre ``intelligent'' comme le Vega pour tester le paramétrage du vario.
\end{itemize}

\subsection*{\xc{} avec un simulateur de vol}
Une bonne manière d'apprendre à se servir du logiciel est de connecter un smartphone à un PC sur lequel tourne un simulateur de vol qui peut envoyer des messages NMEA vers un port série. Condor et X-Plane le permettent par exemple.

Le gros avantage de s'entraîner ainsi est qu'\xc{} en alors utilisé en mode VOL. De ce fait, il se comporte exactement comme si vous voliez. Vous pouvez donc avoir un très bon aperçu du fonctionnement d'\xc{} quand vous utilisez le simulateur de vol. 

\section{Utiliser \xc{} en sécurité}\label{sec:usingxcsoarsafely}\label{conf:usingxcsoarsafely}
L'utilisation en vol d'un calculateur tel qu'\xc{} entraîne certains risques~: l'attention du pilote à son environnement peut être diminuée de manière très significative, de même que le temps passé à regarder en dehors du cockpit pour assurer la sécurité.

La philosophie guidant la conception et le développement d'\xc{} est de réduire cette distraction en minimisant les interactions de l'utilisateur et en présentant les informations de façon claire et lisible du premier coup d'œil.

Les pilotes qui utilisent \xc sont responsables de son utilisation en sécurité. 
Pour bien utiliser \xc{} vous devez~:
\begin{itemize}
\item être à l'aise avec l'utilisation du logiciel par un entraînement au sol.
\item en vol, prendre l'habitude de regarder dehors autour de vous avant d'interagir avec le logiciel, et vous assurer ainsi qu'il n'y a pas de risques de collision avec d'autres aéronefs.
\item configurer le logiciel pour profiter des fonctionnalités automatisées et minimiser ainsi les interactions avec le logiciel. Si vous vous apercevez que vous répétez fréquemment les mêmes successions d'actions, demandez-vous (ou demandez à un autre utilisateur d'\xc) si le logiciel ne peut pas être configuré pour le faire à votre place.
\end{itemize}
