This describes the setup of a development environment suitable to compile \xc
for most supported platforms. The manual focuses on recent releases of
Debian-based flavors of GNU/Linux (including Ubuntu).

In the following instructions, \texttt{sudo} is used to execute commands with
root privileges. This is not enabled by default in Debian (but on some Debian
based distributions, like Ubuntu).

To install a virtual machine with the required, you can use \texttt{Vagrant},
see section \ref{sec:vagrant}.

\section{Download source code}
To download the \xc source code, make sure you have git installed:
\begin{verbatim}
sudo apt-get update
sudo apt-get install git
\end{verbatim}

Download the source code of \xc by executing \texttt{git} in the following way in
your project directory:
\begin{verbatim}
git clone --recurse-submodules git://github.com/XCSoar/XCSoar
\end{verbatim}

\section{Use provisioning scripts}
If you are not using Vagrant, but an existing standard installation of a
Debian-based Linux distribution, you can run the scripts from
\texttt{ide/provisioning} subfolder of the \xc source to install the build
dependencies for various \xc target platforms.

\begin{verbatim}
cd ide/provisioning
sudo ./install-debian-packages.sh
./install-android-tools.sh
\end{verbatim}

\section{Optional: Eclipse IDE}
One of the most widespread IDEs is eclipse. It is not limited to Android, and can be used for all targets. It is not required for \xc, but its installation is described here as an example. Eclipse is quite heavyweight, and many developers prefer other IDEs for \xc development.

To install, download the eclipse installer (Sometimes called ``\emph{Ooomph!}'' for some reason) from here:\\
\texttt{https://www.eclipse.org/downloads/}

Important: Install the CDT version of eclipse for C development, not the Android/Java package, even if you plan developing for Android. In addition, it is very convenient to install the git support (egit).

The current stable version is \emph{eclipse mars} (4.5) and works with OpenJDK 7 or 8, the new \emph{eclipse neon} 4.6, currently RC2, is also quite stable, and requires OpenJDK 8. Both can be installed with the installer.

You can also install the ADT (Android development tools) package for better integration with Android.

Next, create a new project, by generating a make project from existing sources files. Choose your xcsoar source directory which contains the makefile.

Important: After you have added the sources, eclipse will start indexing all files. If you have already started \texttt{make} before this time, then a lot of files have been downloaded for the various libraries which are exctracted/built within the \xc directory (most notably the boost libraries). Indexing all these takes a very long time, and a lot of heap space, so you should probably stop the indexer right away. In addition you should probably exclude these directories from the indexer for the future.

For this, in the C/C++ scope, right-click on the ``output'' directory in the file tree on the left side, select ``Properties'', then ``Resource/Resource Filters'' and add a filter. In the ``add filter'' dialog, choose ``exclude all'', ``files and folders'', ``all children (recursive)'' and set the Filter details to ``Name matches *''.
This will exclude the output tree from the indexer, leading to a minimal index.

\section{Optional: modern LaTeX editor for editing the Manual}
Most people today edit LaTeX files in specific editors, as this is much more comfortable and efficient. This is highly recommended especially if you are not very familiar with LaTeX: learning it is very easy with a modern editor. Here, we install TeXstudio as an example, as it is very widespread and supports the rather rare LuaLaTeX well.

To install, get the relevant package:

\begin{verbatim}
sudo apt-get install texstudio
\end{verbatim}

As the directory tree of \xc is very unusual for a LaTeX project, we need to make some special configurations in order to allow for quick compiling from within the editor, and for full synctex functionality:

In ``Options / Configure TeXStudio'', enable ``show advanced options''.

In ``Options / Configure TeXStudio / Commands / Commands / LuaLaTeX'', replace
\begin{verbatim}
lualatex -synctex=1 -interaction=nonstopmode %.tex
\end{verbatim}
with
\begin{verbatim}
lualatex -synctex=1 -interaction=nonstopmode
 -output-directory=?a)../../../output/manual %.tex
\end{verbatim}

In ``Options / Configure TexStudio / Build / Build Options / Addition Search Paths'':\\
Enter in \emph{both} fields (``Log file'' and in the field ``PDF File''):
\begin{verbatim}
../../../output/manual/
\end{verbatim}


Add the following line to \emph{both} the \texttt{.profile} and the \texttt{.bashrc} file of your user directory:
\begin{maxipage}
\begin{verbatim}
export TEXINPUTS="..:../../../output/manual:../../../output/manual/en:../../..:"
\end{verbatim}
\end{maxipage}


Finally, you need to run ``make manual'' in the \xc base directory at least once from the command line before you can compile from within the TexStudio interface. This creates the path structure and generates the figure files which are included into the manual. Of course, if you change figures, you might have to run ``make manual'' again.

Inside TeXStudio, open the file ``XCSoar-manual.tex'' (or one of the other root files) and right-click on this file to ``set as explicit root document'', in the structure view on the left. Now you are good to go. Make changes and press F5 to see the result immediately.
