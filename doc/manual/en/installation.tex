\chapter{Installation}\label{cha:installation}

\section{To run XCSoar}
 you need to obtain the following:

\begin{itemize}
\itemsep0em
\item a device to run XCSoar on
\item XCSoar
\item a GPS receiver
\item a waypoint file
\item an airspace file (optional)
\item a map file (optional)
\end{itemize}

\section{! Before you go out flying the first time with XCSoar !}

After having installed XCSoar successfully, you might use it instantly. XCSoar 
will launch with a ready-to-use pre-configuration. But be aware, that up to 
this point your new toy will give you a moving map display only.
\warning \emph{Don't trust the computations.} You have to tell XCSoar which 
plane you are flying in advance.  This is done by specifying your plane's data 
as are polar, weight and some other data.  However it's always a good idea to 
study the manual and become familiar with XCSoar at home.

\section{How to get the most from XCSoar}

In order to achieve your maximum benefit from XCSoar, you are kindly asked to 
do some more after having installed the software and downloaded some data
files. ``Some more'' includes personal and plane data as well as configuration 
of just a few features. Unless you are willing to tweak everything XCSoar 
provides, this could be done in a rather short time. Necessary steps are 
summarized in an \emph{XCSoar Checklist}, given in the next section.

If you are planning to arrange a system with several components engaged in, 
this manual will give you valuable advice on both, how to set up things and how 
to use them.

If you are a pilot in a hurry, the authors suggest to change to the short form 
manual \texttt{XCSoar-in-a-Flash} going through the \emph{XCSoar Checklist} 
step by step. The short form manual is available on \url{http://www.xcsoar.org/
discover/manual.html}.



\section{XCSoar Checklist}

\subsection*{{Bring XCSoar into play}}
\begin{itemize}
\item get hardware and install XCSoar
\item get appropriate data files of your flight district
\item tell XCSoar which data files to use
\item tell XCSoar about the glider's polar \& weight
\item possibly connect to instruments
\item finish setup and familiarize
\item mount hardware
\item add items listed underneath to your checklists
\item make ``home''
\end{itemize}

\subsection*{Conduct preflight check, including}
\begin{itemize}
\item setup polar and weight
\item setup wind and flight parameters (MC, bugs, QNH)
\item possibly setup a task
\end{itemize}

\subsection*{Conduct start check, including}
\begin{itemize}
\item Check wind and flight setup once more
\end{itemize}
\vspace{2em}

\subsection*{Fly, enjoy}
\vspace{4em}

\subsection*{Conduct after flight check}
\begin{itemize}
\item Download flight logs from logger, upload to skylines and OLC
\item Gather statistical data of flight.
\end{itemize}
\newpage




\section{Compatibility}

\subsection*{Devices for running XCSoar}

XCSoar runs on the following platforms:

\begin{itemize}
\item Android mobile phones and tablets with Android 1.6 or newer \\
  Example: Dell Streak, Samsung Galaxy S II, HTC Desire HD,
  Motorola Xoom
\item eReader Kobo
\item Windows 2000 or newer
\item Linux
\item Mac OS X (outdated)
\end{itemize}

\subsection*{GPS, Logger, Vario}

XCSoar is compatible with any GPS emitting NMEA data.  Most modern
Android devices have a built-in GPS, but for several reasons it is favorable to
connect to one or more external devices:

\begin{itemize}
\item a specialized GPS receiver gains much better reception providing much 
better data for measures and calculations
\item an airspeed indicator allows quick and exact wind estimates
  without circling
\item a vario improves the thermal assistant
\item a task can be declared to an IGC logger, and after landing, the
  flight log can be downloaded
\item some varios allow synchronising the MacCready setting with
  XCSoar
\item FLARM (and even ADS-B input) provide informations and states of others
around you (and of course FLARM gives you collision detection)
\end{itemize}

\subsection*{Supported external devices and features}
\label{sec:supported-varios}

\newcommand{\y}[0]{{ $\surd$ }}
%{0.8\textwidth}
\noindent\makebox[\textwidth]{%
\begin{tabular}{l|ccc|cc|cc|c}
       \multicolumn{1}{r}{Supported:} & \multicolumn{3}{c|}{-Features} & \multicolumn{5}{c}{-Stream Data} \\
NMEA Device & 
  \begin{sideways} Declaration\end{sideways} & 
  \begin{sideways} Remote ctrl.\end{sideways} & 
  \begin{sideways} Download\end{sideways} &
  \begin{sideways} Airspeed\end{sideways} & 
  \begin{sideways} Vario\end{sideways} & 
  \begin{sideways} Baro.\ altitude\end{sideways} &
  \begin{sideways} Wind\end{sideways} &
  \begin{sideways} G-Sensor\end{sideways} \\
\hline
\input{supported_devices_table.tex}
\end{tabular}}

While most Windows CE based devices have a serial port, such legacy
hardware is not present in modern Android devices.  Those can either
use Bluetooth or the Android IOIO board.  To use Bluetooth, you need
to connect the external device to a Bluetooth-to-Serial adapter, such
as the K6-Bt or the SoarTronic-BT1/2. %Glidertools is defunct.


\section{Software installation}

The software is available as a free download from the XCSoar website
~\xcsoarwebsite{}.  This section describes which file should be
downloaded, and how to install it.

\subsection*{On Android}

Obtain XCSoar from Google's Android market, or install the \verb|apk|
file manually.  Copy the data files on the SD card in the directory
\verb|XCSoarData|.

\subsection*{On a Kobo Mini}

The Kobo Mini is a cheap e-book reader.  It has a black and white
e-paper screen with excellent sunlight readability.

Before you begin: back up the internal SD card.  The XCSoar installer
may break the Kobo, though unlikely.  You can always recover the Kobo
from a software failure, but only if you have access to a backup.

To install XCSoar, connect the Kobo to your PC via USB.  The Kobo
exposes a mass storage device to your PC; open it and create a
directory called \texttt{.kobo} (note the leading dot).  Download the
file \texttt{KoboRoot.tgz} from the XCSoar website into that
directory (\url{http://www.xcsoar.org/hardware/}). Unplug the Kobo and reboot it (switch it off completely,
and switch it on again).  You will see the message ``Updating'', and
after a few minutes, the Kobo shows a menu that allows you to launch
XCSoar or the Kobo \mbox{e-book} reader software.

To copy data files (maps, waypoints, ...) to the Kobo, launch the
original Kobo software (``Nickel'') and connect the Kobo to your PC
again.  Copy the files into a directory called \texttt{XCSoarData} at
the top level.

Alternatively data files may be downloaded via XCSoar's file manager,
after having invoked a WLAN connection just before launching XCSoar.

\subsubsection{Hacking the Kobo}

After installing XCSoar on the Kobo, the new boot script checks for a
script called \texttt{XCSoarData/kobo/init.sh} and executes it.  If
you know what you're doing, you can use this script to do things at
boot time, for example launch \texttt{inetd} (for \texttt{telnet}
access).

When you launch \texttt{Nickel} (the original e-book firmware), the new
boot script also checks for a script in \texttt{XCSoarData/kobo/} named
\texttt{init\_nickel.sh} and executes it. Again, if
you know what you're doing, you can use this script to do things
before \texttt{Nickel} is fully started, for example send commands
to your attached vario (to shut it down, to cut the volume, etc...).

\subsection*{On a Windows PC}

Download the program file \verb|XCSoar.exe| (target ``PC'') to your
hard disk.

\subsection*{On Unix/Linux}

The file downloaded is \verb|xcsoar_XXX.deb|, where \verb|XXX| includes
the version number and platform, e.g. \verb|xcsoar_6.0.4_i386.deb|.
This is a Debian package and can be installed using
\begin{center}
\verb|sudo dpkg -i xcsoar_XXX.deb|.
\end{center}
Use \verb|dpkg-query -L xcsoar| to see where the executable and 
other files are installed,
Additional data files must be placed in the directory
\verb|~/.xcsoar|.
If \verb|~/.xcsoar| does not exist, it will be created the first time
that \verb|xcsoar| is run.

\subsection*{On Raspberry Pi and Cubieboard}

Install the Debian package as described above.  However, unlike on
``regular'' Linux, XCSoar will not use X11.  Instead, it runs
full-screen on the Linux console.

XCSoar requires access to your input devices
(\texttt{/dev/input/event*}).  By default, only \texttt{root} is
granted access.  To override this, create a \texttt{udev}
configuration file, e.g. \texttt{/etc/udev/rules.d/99-input.rules}:

\begin{verbatim*}
KERNEL=="event*", NAME="input/%k", MODE="660", GROUP="input"
\end{verbatim*}

Create the group \texttt{input} and make your user a member:

\begin{verbatim*}
groupadd input
adduser pi input
\end{verbatim*}

\section{Data files}\label{sec:data files}

To be able to use XCSoar's advanced features, additional data files, such as
terrain, topography, special use airspace, waypoints etc.\ are needed. The files
that can be used with XCSoar are described in Chapter~\ref{cha:data-files}.

All data files should be copied into the directory
\texttt{XCSoarData}.  This directory must be in a specific place
so that XCSoar knows where to look for data files:

\begin{description}
\item[Windows PC]
\texttt{XCSoarData} is in your personal folder (``\texttt{My
Documents}'')
\item[Windows Mobile PDA/PNA]
If there is a directory named \texttt{XCSoarData} in the same
directory as \texttt{XCSoar.exe}, then this one will be used.
\texttt{XCSoarData} is on the SD card.  If there is no SD card, then
XCSoar looks for it in \texttt{My Documents}.
\item[Unix/Linux]
The directory is called \verb|.xcsoar| in the user's home directory.
\item[Android Devices]
\texttt{XCSoarData} is on the SD card.
\end{description}


XCSoar will generate a number of additional files at run time.  These
will be placed in the  \texttt{XCSoarData} directory (Windows PC, 
Windows and Android mobile devices), or the \texttt{.xcsoar} directory (Unix/Linux
PC).  From the first run, XCSoar will create and maintain the files
\texttt{user.cup} (user-edited waypoints),
\texttt{Default.tsk} (Default Task),  
\texttt{default.prf} 
(configuration settings),
\texttt{xcsoar.log}, 
plus three directories: \texttt{cache},
\texttt{config} and \texttt{logs}.  Additional files may be
created/modified while XCSoar is running, such as task files
(\texttt{*.tsk}) and flight logs.


\section{Running XCSoar}
%\subsection*{Fly and simulator modes}

Two modes are available inside the XCSoar application: 
\begin{description}
\item[FLY] This mode is used when actually flying.  The simulator is 
  disabled and serial communications are active. 
\item[SIM] This starts XCSoar in simulator mode, no serial communications
  are attempted.
\end{description}

\subsection*{XCSoar PC version}
The program can be run by opening the explorer window, finding the directory
that has the XCSoar.exe executable, and double clicking on that program file.

The program command line options allows the screen orientation of
the display to be defined:
\begin{description}
\item[-portrait] The screen is 480 pixels wide, 640 pixels high.
\item[-square] The screen is 480 pixels wide, 480 pixels high.
\item[-landscape] The screen is 640 pixels wide, 480 pixels high. This is the
usual setting. If you don't specify this parameter the landscape version will be
loaded automatically.
\item[-small] Draws the screen at half size.  This is useful for using XCSoar in
 conjunction with flight simulators e.g.\ Condor.
\end{description}
To change the screen orientation, it is convenient to create a shortcut to the
program, then right click on the shortcut icon and click on ``Shortcut''. 
In the field ``Target'' add one of the desired options listed above.

\subsection*{XCSoar Unix/Linux PC version}
Run \verb|xcsoar| from a command line, or create a shortcut on the
desktop.  The location of the executable file may be found using
\verb|which xcsoar|.  Only landscape mode is  supported for now.

\subsection*{Loading data files}\label{sec:loaddatafiles}
The first time that XCSoar is run, it does not automatically load the 
data files that you placed in the \verb|XCSoarData| directory.  
To tell XCSoar which files to load, double click/tap the map (the large,
blank white part with the glider symbol in the center),
choose the menu \bmenug{Config 2} (click/tap it twice), then select the item 
\mbox{\bmenug{System}.}  The Configuration screen should be displayed:
\sketch{figures/config-basic.png}
The first page allows you to choose the map, 
waypoint and airspace files, by clicking/tapping on the text boxes.
Many other features of XCSoar may be configured here. These are described in detail in Chapter
\ref{cha:configuration}.
Once completed, XCSoar reloads those files; from now on, the data files
will be loaded automatically at run time.

\subsection*{Start-up and user profiles}\label{sec:profiles}
When XCSoar starts up, it will check for existing profiles. If multiple
profiles are detected it will display a small window asking you which profile
to load. To proceed, choose the desired profile and press Enter. If no
profile is chosen the settings from the last session are loaded again. Profiles
can be useful for example in the following cases:
\begin{itemize}
\item Different pilots
\item Competition versus casual flying
\item Flying in different locations
\item Different planes (with different polars)
\end{itemize}
Profiles also might be understood as preserving a backup of a certain 
configuration. Virtually every of XCSoar's settings is stored in a profile 
file with extension \texttt{.prf}. Once you are happy with all your settings, 
make two copies of your profile file. One carrying extension \texttt{.prf}, 
the second copy carrying extension \texttt{.bak}.
Whereas the \texttt{.prf} file will show up on startup and reflect all of 
your changes made whilst XCSoar is running for the next startup, the file 
\texttt{.bak} will keep the settings, you judged being worth it to preserve. 
As an example you might create a set of files as is:
\begin{itemize}
\item \texttt{buddiesinArcus.prf}
\item \texttt{buddiesinArcus.bak}
\item \texttt{johninKa6atwonderland.prf}
\item \texttt{johninKa6atwonderland.bak}
\end{itemize}

\subsection*{SIM mode}
XCSoar comes with a simple interface allowing for conducting a simulated 
flight. Depending on the hardware platform, there are different 
methods for altering values of bearing, speed, and height.  Simulation is
intended for your first familiarization with XCSoar in action. If you like the 
idea simulating a more realistic thing at home, you should acquire a ``true''
soaring simulator, to be connected to XCSoar.

On the map screen, clicking/touching the glider symbol
with touchscreen or mouse and dragging 
causes the glider to move in the direction of the drag, the
speed being proportional to the length of the drag.

With buttons, the aircraft
speed, altitude and direction may be changed using the Infoboxes.
The following may not be available in total on every hardware platform, but on 
any of the platforms XCSoar supports, a full set of inputs needed for 
simulation purposes is possible.

By pushing an Infobox, you select a value to be altered by buttons, either 
 hardware or menu buttons.
The aircraft altitude can be adjusted by selecting the GPS altitude
InfoBox \bmenuw{Alt GPS}, and touching the up or down key or buttons on the 
touchscreen.
The airspeed can be adjusted by selecting the Infobox ground speed
\bmenuw{V Gnd}, touching up or down key or menu buttons.
The glider's track  can be adjusted by selecting the track InfoBox
\bmenuw{Track}, and touching up/down buttons.

With either of the \bmenuw{Alt GPS} or \bmenuw{V Gnd}
selected, the glider's direction may be changed using the left/right keys.

Other controls, buttons and menus work the same as in FLY mode.


\subsection*{Splash screen}
When XCSoar starts up, shuts down, or loads large files, such as airspace,
waypoints, terrain, etc., a progress screen is presented while the data is being
loaded. This screen has a progress bar which indicates the data loading
activity, and a short line of text describing the action that is being performed.

This screen also displays the software version information.

\subsection*{Exiting the program}
For PDA and PC versions, XCSoar is shut down from the menu. The menu can be
opened by double-clicking on the map or the \InfoBox es.
\begin{quote}
\bmenug{QUIT}
\end{quote}

For PC versions, XCSoar can also be shut down by clicking the close icon
on the XCSoar window.
